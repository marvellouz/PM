\documentclass{beamer}
\usepackage[utf8]{inputenc}
\usepackage[T1]{fontenc}
\usepackage[bulgarian]{babel}
\usepackage{alltt}
\usepackage{textcomp}
\useoutertheme{shadow}

\setbeamercolor{title}{fg=red!80!black}
\setbeamercolor{frametitle}{fg=red!80!black}

\usetheme[secheader]{Madrid}
\usecolortheme{crane}

%Тема 11.
%7.2 Constraints on Attributes and Tuples
%• Not-Null Constraints
%• Attribute-Based CHECK Constraints
%• Tuple-Based CHECK Constraints
%• Modification of Constraints
%• Giving Names to Constraints
%• Altering Constraints on Tables

\title[Учредяване на проект за Информационна система за телекомуникационна компания]{Учредяване на проект \\ за \\ Информационна система за телекомуникационна компания}
\author{Емил Станчев 71100,\\
Валентина Динкова 71112,\\
Ивайло Михайлов 71102,\\
Николай Варадинов 71122,\\
Мария Григорова 71058,\\
Станислав Трифонов 71094}
\institute{ФМИ}
\date{\today}
\begin{document}
\begin{frame}
  \titlepage
\end{frame}



\begin{frame}
  \frametitle{Описание на проекта}
Изграждане и внедряване на цялостна иформационна система с web интерфейс за вътрешно ползване 
от служителите и ръководството на телекомуникационна Компания. 
\newline
\newline
\textbf{Инициатор:} г-н Любомир Нейков, управител на Компанията
\textbf{Ръководител:} г-н Димитър Телкийски
\end{frame}


\begin{frame}
  \frametitle{Предназначение на проекта}
Проектът има за цел да подобри  работните процеси във фирмата, да улесни работата на служителите
и да подобри работата с клиенти. 
\end{frame}

\begin{frame}
  \frametitle{Измерими цели на проекта}
Проектът има за цел да улесни работата на служителите във фирмата и да подобри работата с клиенти. 
Бързодействието на служителите следва да се повиши значиелно, а времето за извеждане на справки се 
намалява със 70 \textdiscount .
\newline
\newline
Внедряване на системата във фирмата и обучение на 25 потребители от отделите на компанията.
\end{frame}

\begin{frame}
  \frametitle{Изисквания към проекта}
Системата трябва да поддържа около 200 паралелни заявки към всяка една своя част, като 
всяка заявка да отнема най-много 10 секунди, за да не забавя работата на служителите.
Своевременно остраняване на възникнали проблеми от страна на „Октопод“ ООД.
Системата трбява да е достъпна само във вътрешната мрежа на компанията. 
Извън работно време  и извън мрежата на компанията, достъпът се разрешава от администратора. 
\end{frame}

\begin{frame}
  \frametitle{Времеви рамки}
\textbf{Проектиране на системата} – 4 седмици \\
\textbf{Изграждане на системата} – 25 седмици \\
\textbf{Финално тестване и отстраняване на проблеми} – 6 седмици \\
\textbf{Инсталиране на системата} – 1 седмица \\
\textbf{Обучение на потребители} – 2 седмици
\newline
\newline
\textbf{Общо:} 38 седмици 

\end{frame}

\begin{frame}
  \frametitle{Рискове}
Некоректност или невъзможност за изпълнение на договорните задължения от страна на фирмата възложител.
Възникване на трудности при обучението на персонала – удължаване на срока на обучение.
Нежелание за работа със системата от страна на потребителите.

\end{frame}


\begin{frame}
  \frametitle{Заинтересовани лица}
\begin{itemize}
 \item \textbf{Вътрешни:}
Ръководител на проекта, управител на фирмата, консултант (обучаващ потребителите на системата)
  \item \textbf{Външни:}
Управител на Компанията, потребител, администратор, изпълнител на счетоводния модул, клиенти на Компанията, конкурентни фирми
\end{itemize}
\end{frame}

\begin{frame}
  \frametitle{}

\end{frame}






\end{document}
